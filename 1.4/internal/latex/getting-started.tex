Internal DocumentationOther\+: \href{../user/getting-started.html}{\tt User} -\/ \href{../developer/getting-started.html}{\tt Developer}

The first step is to install the W\+R\+E\+N\+CH library, following the instructions on the \hyperlink{install}{installation page}.\hypertarget{getting-started_getting-started-example}{}\section{Running a First Example}\label{getting-started_getting-started-example}
Typing {\ttfamily make} in the top-\/level directory will compile the examples, and {\ttfamily make install} will put the examples binaries in the {\ttfamily /usr/local/bin} folder (for Mac\+OS and most Linux distributions).

W\+R\+E\+N\+CH provides a simple example in the {\ttfamily examples/simple-\/example} directory, which generates two executables\+: a cloud-\/based example {\ttfamily wrench-\/simple-\/example-\/cloud}, and a batch-\/system-\/based (e.\+g., S\+L\+U\+RM) example {\ttfamily wrench-\/simple-\/example-\/batch}. To run the examples, simply use one of the following commands\+:


\begin{DoxyCode}
# Runs the cloud-based implementation
wrench-simple-example-cloud \(\backslash\)
    <PATH-TO-WRENCH-FOLDER>/examples/simple-example/platform\_files/cloud\_hosts.xml \(\backslash\)
    <PATH-TO-WRENCH-FOLDER>/examples/simple-example/workflow\_files/genome.dax

# Runs the batch-based implementation
wrench-simple-example-batch \(\backslash\)
    <PATH-TO-WRENCH-FOLDER>/examples/simple-example/platform\_files/batch\_hosts.xml \(\backslash\)
    <PATH-TO-WRENCH-FOLDER>/examples/simple-example/workflow\_files/genome.dax
\end{DoxyCode}
\hypertarget{getting-started_getting-started-example-simple}{}\subsection{Understanding the Simple Example}\label{getting-started_getting-started-example-simple}
Both versions of the example (cloud of batch) require two command-\/line arguments\+: (1) a \href{http://simgrid.gforge.inria.fr/simgrid/3.19/doc/platform.html}{\tt Sim\+Grid virtual platform description file}; and (2) a W\+R\+E\+N\+CH workflow file.


\begin{DoxyItemize}
\item {\bfseries Sim\+Grid simulated platform description file\+:} A \href{http://simgrid.gforge.inria.fr}{\tt Sim\+Grid} simulation must be provided with the description of the platform on which an application execution is to be simulated. This is done via a platform description file, in X\+ML, that includes definitions of compute hosts, clusters of hosts, storage resources, network links, routes between hosts, etc. A detailed description on how to create a platform description file can be found \href{http://simgrid.gforge.inria.fr/simgrid/3.19/doc/platform.html}{\tt here}.
\item {\bfseries W\+R\+E\+N\+CH workflow file\+:} W\+R\+E\+N\+CH provides native parsers for \href{http://workflowarchive.org}{\tt D\+AX} (D\+AG in X\+ML) and \href{https://github.com/wrench-project/wrench/tree/master/doc/schemas}{\tt J\+S\+ON} workflow description file formats. Refer to their respective Web sites for detailed documentation.
\end{DoxyItemize}

The source file for the cloud-\/based simulator is at {\ttfamily examples/simple-\/example/\+Simulator\+Cloud.\+cpp} and at {\ttfamily examples/simple-\/example/\+Simulator\+Batch.\+cpp} for the batch-\/based example. These source files, which are heavily commented, and perform the following\+:


\begin{DoxyItemize}
\item The first step is to read and parse the workflow and the platform files, and to create a simulation object ({\ttfamily \hyperlink{classwrench_1_1_simulation}{wrench\+::\+Simulation}}).
\item A storage service ({\ttfamily \hyperlink{classwrench_1_1_simple_storage_service}{wrench\+::\+Simple\+Storage\+Service}}) is created and deployed on a host.
\item A cloud ({\ttfamily \hyperlink{classwrench_1_1_cloud_service}{wrench\+::\+Cloud\+Service}}) or a batch ({\ttfamily \hyperlink{classwrench_1_1_batch_service}{wrench\+::\+Batch\+Service}}) service is created and deployed on a host. Both services are seen by the simulation as compute services ({\ttfamily \hyperlink{classwrench_1_1_compute_service}{wrench\+::\+Compute\+Service}}) – jobs can then be submitted to these services.
\item A Workflow Management System ({\ttfamily \hyperlink{classwrench_1_1_w_m_s}{wrench\+::\+W\+MS}}) is instantiated (in this case the {\ttfamily Simple\+W\+MS}) with a reference to a workflow object ({\ttfamily \hyperlink{classwrench_1_1_workflow}{wrench\+::\+Workflow}}) and a scheduler ({\ttfamily wrench\+::\+Scheduler}). The scheduler implements the decision-\/making algorithms inside the W\+MS. These algorithms are modularized (so that the same W\+MS implementation can be iniated with various decision-\/making algorithms in different simulations). The source codes for the schedulers, which is of interest to \char`\"{}\+Developers\char`\"{} (i.\+e., those users who use the W\+R\+E\+N\+CH Developer A\+PI), is in directory {\ttfamily examples/scheduler}.
\item A file registry ({\ttfamily \hyperlink{classwrench_1_1_file_registry_service}{wrench\+::\+File\+Registry\+Service}}), a.\+k.\+a. a file replica catalog, which keeps track of files stored in different storage services, is deployed on a host.
\item Workflow input files are staged on the storage service
\item The simulation is launched, executes, and completes.
\item Timestamps can be retrieved to analyze the simulated execution.
\end{DoxyItemize}

This simple example can be used as a blueprint for starting a large W\+R\+E\+N\+C\+H-\/based simulation project. The next section provides further details about this process.\hypertarget{getting-started_getting-started-prep}{}\section{Preparing the Environment}\label{getting-started_getting-started-prep}
Internal developers are expected to {\bfseries contribute} code to W\+R\+E\+N\+CH\textquotesingle{}s core components. Please, refer to the \href{./annotated.html}{\tt A\+PI Reference} to find the detailed documentation for W\+R\+E\+N\+CH functions.

\begin{quote}
{\bfseries Note\+:} It is strongly recommended that W\+R\+E\+N\+CH internal developers (contributors) {\itshape fork} W\+R\+E\+N\+CH\textquotesingle{}s code from the \href{http://github.com/wrench-project/wrench}{\tt Git\+Hub repository}, and create pull requests with their proposed modifications. \end{quote}
\hypertarget{getting-started_getting-started-structure}{}\section{W\+R\+E\+N\+C\+H Directory and File Structure}\label{getting-started_getting-started-structure}
W\+R\+E\+N\+CH follows a standard C++ project directory and files structure\+:


\begin{DoxyCode}
.
+-- doc                        # Documentation source files
+-- docs                       # Generated documentation files
+-- examples                   # Examples folder (includes workflows, platform files, and implementations) 
+-- include                    # WRENCH header files - .h files 
+-- src                        # WRENCH source files - .cpp files
+-- test                       # WRENCH test files
+-- tools                      # Tools for supporting documentation generation and release builds
+-- .travis.yml                # Configuration file for Travis Continuous Integration
+-- sonar-project.properties   # Configuration file for Sonar Cloud Continuous Code Quality
+-- LICENSE.md                 # WRENCH license disclaimer
+-- README.md
\end{DoxyCode}
 