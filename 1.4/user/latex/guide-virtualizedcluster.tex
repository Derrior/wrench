User DocumentationOther\+: \href{../developer/guide-virtualizedcluster.html}{\tt Developer} -\/ \href{../internal/guide-virtualizedcluster.html}{\tt Internal}\hypertarget{guide-virtualizedcluster_guide-virtualizedcluster-overview}{}\section{Overview}\label{guide-virtualizedcluster_guide-virtualizedcluster-overview}
A virtualized cluster service is an abstraction of a compute service that corresponds to a platform of physical resources on which Virtual Machine (VM) instances can be created. The virtualized cluster service provides all necessary abstractions to manage V\+Ms and their allocation to physical resources, including creation on specific physical hosts, suspend/resume, migration, etc. Compute jobs submitted to the virtualized cluster service run on previously created VM instances. If a VM that meets a job\textquotesingle{}s requirements cannot be found, the service will throw an exception. In the virtualized cluster service, a VM behaves as a \hyperlink{guide-baremetal}{bare-\/metal} service.

The main difference between a \hyperlink{guide-cloud}{cloud service} and a virtualized cluster service is that the latter does expose the underlying physical infrastructure (e.\+g., it is possible to instantiate a VM on a particular physical host, or to migrate a VM between two particular physical hosts).\hypertarget{guide-virtualizedcluster_guide-virtualizedcluster-creating}{}\section{Creating a virtualized cluster compute service}\label{guide-virtualizedcluster_guide-virtualizedcluster-creating}
In W\+R\+E\+N\+CH, a virtualized cluster service represents a cloud service (\hyperlink{classwrench_1_1_cloud_service}{wrench\+::\+Cloud\+Service}, which itself represents a \hyperlink{classwrench_1_1_compute_service}{wrench\+::\+Compute\+Service}), which is defined by the \hyperlink{classwrench_1_1_virtualized_cluster_service}{wrench\+::\+Virtualized\+Cluster\+Service} class. An instantiation of a virtualized cluster service requires the following parameters\+:


\begin{DoxyItemize}
\item A hostname on which to start the service (this is the entry point to the service)
\item A list ({\ttfamily std\+::vector}) of hostnames (all cores and all R\+AM of each host is available to the virtualized cluster service)
\item A scratch space size, i.\+e., the size in bytes of storage local to the virtualized cluster service (used to store workflow files, as needed, during job executions)
\item Maps ({\ttfamily std\+::map}) of configurable properties ({\ttfamily \hyperlink{classwrench_1_1_virtualized_cluster_service_property}{wrench\+::\+Virtualized\+Cluster\+Service\+Property}}) and configurable message payloads ({\ttfamily \hyperlink{classwrench_1_1_virtualized_cluster_service_message_payload}{wrench\+::\+Virtualized\+Cluster\+Service\+Message\+Payload}}).
\end{DoxyItemize}

The example below shows how to create an instance of a virtualized cluster service that runs on host \char`\"{}vc\+\_\+gateway\char`\"{}, provides access to 4 execution hosts, and has a scratch space of 1\+TiB\+:


\begin{DoxyCode}
\textcolor{keyword}{auto} virtualized\_cluster\_cs = simulation.add(
          \textcolor{keyword}{new} \hyperlink{classwrench_1_1_virtualized_cluster_service}{wrench::VirtualizedClusterService}(\textcolor{stringliteral}{"vc\_gateway"}, \{\textcolor{stringliteral}{"host1"}, \textcolor{stringliteral}{"
      host2"}, \textcolor{stringliteral}{"host3"}, \textcolor{stringliteral}{"host4"}\}, pow(2,40),
                                                \{\{
      \hyperlink{classwrench_1_1_compute_service_property_af0abab1e3bce4932c4482031f0c31ce8}{wrench::VirtualizedClusterServiceProperty::SUPPORTS\_PILOT\_JOBS}
      , \textcolor{stringliteral}{"false"}\}\}));
\end{DoxyCode}
\hypertarget{guide-virtualizedcluster_guide-virtualizedcluster-creating-properties}{}\subsection{Virtualized cluster service properties}\label{guide-virtualizedcluster_guide-virtualizedcluster-creating-properties}
All properties for a virtualized cluster service are inherited from {\ttfamily \hyperlink{classwrench_1_1_cloud_service_property}{wrench\+::\+Cloud\+Service\+Property}} (see the \hyperlink{guide-cloud_guide-cloud-creating-properties}{Cloud service properties} section). 