User DocumentationOther\+: \href{../developer/wrench-101.html}{\tt Developer} -\/ \href{../internal/wrench-101.html}{\tt Internal}

W\+R\+E\+N\+CH 101 is a page and a set of documents that provide detailed information for each W\+R\+E\+N\+CH\textquotesingle{}s \hyperlink{index_overview-users}{classes of users}, and higher-\/level content than the \href{./annotated.html}{\tt A\+PI Reference}. For instructions on how to \hyperlink{install}{install}, run a \hyperlink{getting-started}{first example}, or \hyperlink{getting-started_getting-started-prep}{create a basic W\+R\+E\+N\+C\+H-\/based simulator}, please refer to their respective sections in the documentation.

This {\bfseries User 101} guide describes all the W\+R\+E\+N\+CH simulation components (building blocks) necessary to build a custom simulator and run simulation scenarios.



\hypertarget{wrench-101_wrench-101-simulator-10000ft}{}\section{10,000-\/ft view of a W\+R\+E\+N\+C\+H-\/based simulator}\label{wrench-101_wrench-101-simulator-10000ft}
A W\+R\+E\+N\+C\+H-\/based simulator can be as simple as a single {\ttfamily main()} function that first creates a platform to be simulated (the hardware) and a set of services that run on the platform (the software). These services correspond to software that knows how to store data, perform computation, and many other useful things that real-\/world cyberinfrastructure services can do.

The simulator then needs to create a workflow (or a set of workflows) to be executed, which consists of a set of compute tasks each with input and output files, and thus data-\/dependencies. A special service is then created, called a Workflow Management System (W\+MS), that will be in charge of executing the workflow on the platform. (This service must have been implemented by a W\+R\+E\+N\+CH \char`\"{}developer\char`\"{}, i.\+e., a user that has used the Developer A\+PI). The set of input files to the workflow, if any, are staged on the platform at particular storage locations.

The simulation is then launched via a single call. When this call returns, the W\+MS has cleanly\+\_\+terminated (typically after completing the execution of the workflow, or failing to executed it) and the simulation output can be analyzed.\hypertarget{wrench-101_wrench-101-simulator-blueprint}{}\section{Blueprint for a W\+R\+E\+N\+C\+H-\/based simulator}\label{wrench-101_wrench-101-simulator-blueprint}
Here are the steps that a W\+R\+E\+N\+C\+H-\/based simulator typically follows\+:


\begin{DoxyEnumerate}
\item {\bfseries Create and initialize a simulation} -- In W\+R\+E\+N\+CH, a user simulation is defined via the {\ttfamily \hyperlink{classwrench_1_1_simulation}{wrench\+::\+Simulation}} class. An instance of this class must be created, and the {\ttfamily \hyperlink{classwrench_1_1_simulation_a3c6d35f1f77f35cbc727ce31e5689992}{wrench\+::\+Simulation\+::init()}} method is called to initialize the simulation (and parse W\+R\+E\+N\+C\+H-\/specific and \href{http://simgrid.gforge.inria.fr/simgrid/3.19/doc/options.html}{\tt Sim\+Grid-\/specific} command-\/line arguments). Two useful such arguments are {\ttfamily -\/-\/help-\/wrench}, which displays help messages about optional W\+R\+E\+N\+C\+H-\/specific command-\/line arguments, and {\ttfamily -\/-\/help-\/simgrid}, which displays help messages about optional Simgrid-\/specific command-\/line arguments.
\item {\bfseries Instantiate a simulated platform} -- This is done with the {\ttfamily \hyperlink{classwrench_1_1_simulation_ae22639abf6ede9f345b382f5ffe19b0e}{wrench\+::\+Simulation\+::instantiate\+Platform()}} method which takes as argument a \href{http://simgrid.gforge.inria.fr/simgrid/3.17/doc/platform.html}{\tt Sim\+Grid virtual platform description file}. Any \href{http://simgrid.gforge.inria.fr}{\tt Sim\+Grid} simulation must be provided with the description of the platform on which an application/system execution is to be simulated (compute hosts, clusters of hosts, storage resources, network links, routers, routes between hosts, etc.)
\item {\bfseries Instantiate services on the platform} -- The {\ttfamily \hyperlink{classwrench_1_1_simulation_ad1f5c12285ecfaf5a2ce7dab5ec8b4c5}{wrench\+::\+Simulation\+::add()}} method is used to add services to the simulation. Each class of service is created with a particular constructor, which also specifies host(s) on which the service is to be started. Typical kinds of services include compute services, storage services, network proximity services, and file registry services.
\item {\bfseries Create at least one workflow} -- This is done by creating an instance of the {\ttfamily \hyperlink{classwrench_1_1_workflow}{wrench\+::\+Workflow}} class, which has methods to manually add tasks and files to the workflow application, but also methods to import workflows from standard workflow description files (\href{http://workflowarchive.org}{\tt D\+AX} and \href{https://github.com/wrench-project/wrench/tree/master/doc/schemas}{\tt J\+S\+ON}). If there are input files to the workflow\textquotesingle{}s entry tasks, these must be staged on instantiated storage services.
\item {\bfseries Instantiate at least one W\+MS per workflow} -- At least one of the services instantiated must be a {\ttfamily \hyperlink{classwrench_1_1_w_m_s}{wrench\+::\+W\+MS}} instance, i.\+e., a service that is in charge of executing the workflow, as implemented by a W\+R\+E\+N\+CH \char`\"{}developer\char`\"{} using the \href{../developer/wrench-101.html}{\tt Developer} A\+PI. Associating a workflow to a W\+MS is done via the {\ttfamily \hyperlink{classwrench_1_1_w_m_s_afd2a6ae2f4d792046a6a17d5c0dc313f}{wrench\+::\+W\+M\+S\+::add\+Workflow()}} method.
\item {\bfseries Launch the simulation} -- This is done via the {\ttfamily \hyperlink{classwrench_1_1_simulation_ae9589632de9a2311ed1d7f7747478985}{wrench\+::\+Simulation\+::launch()}} call which first sanity checks the simulation setup and then launches all simulated services, until all W\+MS services have exited (after they have completed or failed to complete workflows).
\item {\bfseries Process simulation output} -- The {\ttfamily \hyperlink{classwrench_1_1_simulation_aff0338aa6831c6ac252cf0673fe68f44}{wrench\+::\+Simulation\+::get\+Output()}} method returns an object that is a collection of time-\/stamped traces of simulation events. These traces can be processed/analyzed at will.
\end{DoxyEnumerate}\hypertarget{wrench-101_wrench-101-simulator-services}{}\section{Available services}\label{wrench-101_wrench-101-simulator-services}
To date, these are the (simulated) services that can be instantiated on the simulated platform\+:


\begin{DoxyItemize}
\item {\bfseries Compute Services} (classes that derive {\ttfamily \hyperlink{classwrench_1_1_compute_service}{wrench\+::\+Compute\+Service}})\+: These are services that know how to compute workflow tasks. These include bare-\/metal servers ({\ttfamily \hyperlink{classwrench_1_1_bare_metal_compute_service}{wrench\+::\+Bare\+Metal\+Compute\+Service}}), cloud platforms ({\ttfamily \hyperlink{classwrench_1_1_cloud_service}{wrench\+::\+Cloud\+Service}}), virtualized cluster platforms ({\ttfamily \hyperlink{classwrench_1_1_virtualized_cluster_service}{wrench\+::\+Virtualized\+Cluster\+Service}}), batch-\/scheduled clusters ({\ttfamily \hyperlink{classwrench_1_1_batch_service}{wrench\+::\+Batch\+Service}}). It is not technically required to instantiate a compute service, but then no workflow task can be executed by the W\+MS.
\item {\bfseries Storage Services} (classes that derive {\ttfamily \hyperlink{classwrench_1_1_storage_service}{wrench\+::\+Storage\+Service}})\+: These are services that know how to store workflow files, which can then be accessed in reading/writing by the compute services when executing tasks that read/write files. It is not technically required to instantiate a storage service, but then no workflow task can have an input or an output file.
\item {\bfseries File Registry Services} (the {\ttfamily \hyperlink{classwrench_1_1_file_registry_service}{wrench\+::\+File\+Registry\+Service}} class)\+: These services, often known as {\itshape replica catalogs}, are simply databases of $<$filename, list of locations$>$ key-\/value pairs of the storage services on which a copies of files are available. They are used during workflow execution to decide where input files for tasks can be acquired. It is not required to instantiate a file registry service, unless the workflow\textquotesingle{}s entry tasks have input files (because in this case these files have to be stored at some storage services before the execution can start, and all file registry service are then automatically made aware of where these files are stored). Note that some W\+MS implementations may complain if no file registry service is available.
\item {\bfseries Network Proximity Services} (the class {\ttfamily \hyperlink{classwrench_1_1_network_proximity_service}{wrench\+::\+Network\+Proximity\+Service}})\+: These are services that monitor the network and maintain a database of host-\/to-\/host network distances. This database can be queried by W\+M\+Ss to make informed decisions, e.\+g., to pick from which storage service a file should be retrieved so as to reduce communication time. Typically, network distances are estimated based on round-\/trip-\/times between hosts. It is not required to instantiate a network proximity service, but some W\+MS implementations may complain if none is available.
\item {\bfseries Workflow Management Systems (W\+M\+Ss)} (classes that derive {\ttfamily \hyperlink{classwrench_1_1_w_m_s}{wrench\+::\+W\+MS}})\+: A workflow management system provides the mechanisms for executing workflow applications, include decision-\/making for optimizing various objectives (the most common one is to minimize workflow execution time). By default, W\+R\+E\+N\+CH does not provide a W\+MS implementation as part of its core components, however a simple implementation ({\ttfamily wrench\+::\+Simple\+W\+MS}) is available in the {\ttfamily examples/simple-\/example} folder. Please, refer to the \href{../developer/wrench-101.html}{\tt Developer 101 Guide} section for further information on how to develop a W\+MS. At least {\bfseries one} W\+MS should be provided for running a simulation. Additional W\+M\+Ss implementations may also be found in the \href{http://wrench-project.org}{\tt W\+R\+E\+N\+CH project website}.
\end{DoxyItemize}\hypertarget{wrench-101_wrench-101-customizing-services}{}\section{Customizing Services}\label{wrench-101_wrench-101-customizing-services}
Each service is customizable by passing to its constructor a {\itshape property list}, i.\+e., a key-\/value map where each key is a property and each value is a string. Each service defines a property class. For instance, the {\ttfamily \hyperlink{classwrench_1_1_service}{wrench\+::\+Service}} class has an associated {\ttfamily \hyperlink{classwrench_1_1_service_property}{wrench\+::\+Service\+Property}} class, the {\ttfamily \hyperlink{classwrench_1_1_compute_service}{wrench\+::\+Compute\+Service}} class has an associated {\ttfamily \hyperlink{classwrench_1_1_compute_service_property}{wrench\+::\+Compute\+Service\+Property}} class, and so on at all levels of the service class hierarchy.

The A\+PI documentation for these property classes explains what each property means, what possible values are, and what default values are. Other properties have more to do with what the service can or should do when in operation. For instance, the {\ttfamily \hyperlink{classwrench_1_1_batch_service_property}{wrench\+::\+Batch\+Service\+Property}} class defines a {\ttfamily \hyperlink{classwrench_1_1_batch_service_property_aceb0af1c33f5ff2da347f54a484ce32e}{wrench\+::\+Batch\+Service\+Property\+::\+B\+A\+T\+C\+H\+\_\+\+S\+C\+H\+E\+D\+U\+L\+I\+N\+G\+\_\+\+A\+L\+G\+O\+R\+I\+T\+HM}} which specifies what scheduling algorithm a batch service should use for prioritizing jobs. All property classes inherit from the {\ttfamily \hyperlink{classwrench_1_1_service_property}{wrench\+::\+Service\+Property}} class, and one can explore that hierarchy to discover all possible (and there are many) service customization opportunities.

Finally, each service exchanges messages on the network with other services (e.\+g., a W\+MS service sends a \char`\"{}do some work\char`\"{} message to a compute service). The size in bytes, or payload, of all messages can be customized similarly to the properties, i.\+e., by passing a key-\/value map to the service\textquotesingle{}s constructor. For instance, the {\ttfamily \hyperlink{classwrench_1_1_service_message_payload}{wrench\+::\+Service\+Message\+Payload}} class defines a {\ttfamily \hyperlink{classwrench_1_1_service_message_payload_a9efb4a6b2c8876e17a3f636ba5ac17f6}{wrench\+::\+Service\+Message\+Payload\+::\+S\+T\+O\+P\+\_\+\+D\+A\+E\+M\+O\+N\+\_\+\+M\+E\+S\+S\+A\+G\+E\+\_\+\+P\+A\+Y\+L\+O\+AD}} property which can be used to customize the size, in bytes, of the control message sent to the service daemon (that is the entry point to the service) to tell it to terminate. Each service class has a corresponding message payload class, and the A\+PI documentation for these message payload classes details all messages whose payload can be customized.\hypertarget{wrench-101_wrench-101-logging}{}\section{Customizing logging}\label{wrench-101_wrench-101-logging}
When running a W\+R\+E\+N\+CH simulator you will notice that there is quite a bit of logging output. While logging output can be useful to inspect visually the way in which the simulation proceeds, it often becomes necessary to disable it. W\+R\+E\+N\+CH\textquotesingle{}s logging system is a thin layer on top of Sim\+Grid\textquotesingle{}s logging system, and as such is controlled via command-\/line arguments. The simple example in {\ttfamily examples/simple-\/example} is executed as follows, assuming the working directory is {\ttfamily examples/simple-\/example}\+:


\begin{DoxyCode}
./\hyperlink{namespacewrench}{wrench}-simple-example-cloud  platform\_files/cloud\_hosts.xml workflow\_files/genome.dax
\end{DoxyCode}


One first way in which to modify logging is to disable colors, which can be useful to redirect output to a file, is to use the {\ttfamily -\/-\/wrench-\/no-\/color} command-\/line option, anywhere in the argument list, for instance\+:


\begin{DoxyCode}
./\hyperlink{namespacewrench}{wrench}-simple-example-cloud  --\hyperlink{namespacewrench}{wrench}-no-color platform\_files/cloud\_hosts.xml workflow\_files/
      genome.dax
\end{DoxyCode}


Disabling all logging is done with the Sim\+Grid option {\ttfamily -\/-\/wrench-\/no-\/log}\+:


\begin{DoxyCode}
./\hyperlink{namespacewrench}{wrench}-simple-example-cloud  --\hyperlink{namespacewrench}{wrench}-no-log platform\_files/cloud\_hosts.xml workflow\_files/
      genome.dax
\end{DoxyCode}


The above {\ttfamily -\/-\/wrench-\/no-\/log} option is a simple wrapper around the sophisticated Simgrid logging capabilities (it is equivalent to the Simgrid argument {\ttfamily -\/-\/log=root.\+threshold\+:critical}). Details on these capabilities are displayed when passing the {\ttfamily -\/-\/help-\/logs} command-\/line argument to your simulator. In a nutshell particular \char`\"{}log categories\char`\"{} can be toggled on and off. Log category names are attached to {\ttfamily $\ast$.cpp} files in the W\+R\+E\+N\+CH and Sim\+Grid code. Using the {\ttfamily -\/-\/help-\/log-\/categories} command-\/line argument shows the entire log category hierarchy. For instance, there is a log category that is called {\ttfamily wms} for the W\+MS, i.\+e., those logging messages in the {\ttfamily wrench\+:W\+MS} class and a log category that is called {\ttfamily simple\+\_\+wms} for logging message in the {\ttfamily wrench\+::\+Simple\+W\+MS} class, which inherits from {\ttfamily \hyperlink{classwrench_1_1_w_m_s}{wrench\+::\+W\+MS}}. These messages are thus logging output produced by the W\+MS in the simple example. They can be enabled while other messages are disabled as follows\+:


\begin{DoxyCode}
./\hyperlink{namespacewrench}{wrench}-simple-example-cloud   platform\_files/cloud\_hosts.xml workflow\_files/genome.dax --log=root.
      threshold:critical --log=simple\_wms.threshold=debug --log=wms.threshold=debug
\end{DoxyCode}


Use the {\ttfamily -\/-\/help-\/logs} option displays information on the way Sim\+Grid logging works. See the \href{http://simgrid.gforge.inria.fr/simgrid/latest/doc/outcomes_logs.html}{\tt full Sim\+Grid logging documentation} for all details.\hypertarget{wrench-101_wrench-101-simulation-output}{}\section{Analyzing Simulation Output}\label{wrench-101_wrench-101-simulation-output}
Once the {\ttfamily \hyperlink{classwrench_1_1_simulation_ae9589632de9a2311ed1d7f7747478985}{wrench\+::\+Simulation\+::launch()}} method has returned, it is possible to process time-\/stamped traces to analyze simulation output. The {\ttfamily \hyperlink{classwrench_1_1_simulation_aff0338aa6831c6ac252cf0673fe68f44}{wrench\+::\+Simulation\+::get\+Output()}} method returns an instance of {\ttfamily \hyperlink{classwrench_1_1_simulation_output}{wrench\+::\+Simulation\+Output}}. This object has a templated {\ttfamily \hyperlink{classwrench_1_1_simulation_output_a1d03324f34db985d0e181e42cf30cd9d}{wrench\+::\+Simulation\+Output\+::get\+Trace()}} method to retrieve traces for various information types. For instance, the call 
\begin{DoxyCode}
simulation.getOutput().getTrace<wrench::SimulationTimestampTaskCompletion>()
\end{DoxyCode}
 returns a vector of time-\/stamped task completion events. The classes that implement time-\/stamped events are all classes named {\ttfamily wrench\+::\+Simulation\+Timestamp\+Something}, where {\ttfamily Something} is pretty self-\/explanatory (e.\+g., {\ttfamily Task\+Completion}).\hypertarget{wrench-101_wrench-101-energy}{}\section{Measuring Energy Consumption}\label{wrench-101_wrench-101-energy}
W\+R\+E\+N\+CH leverages \href{http://simgrid.gforge.inria.fr/simgrid/latest/doc/group__plugin__energy.html}{\tt Sim\+Grid\textquotesingle{}s energy plugin}, which provides accounting for computing time and dissipated energy in the simulated platform. Sim\+Grid\textquotesingle{}s energy plugin requires host {\bfseries pstate} definitions (levels of performance, C\+PU frequency) in the \href{http://simgrid.gforge.inria.fr/simgrid/latest/doc/platform.html}{\tt X\+ML platform description file}. The following is a list of current available information provided by the plugin\+:


\begin{DoxyItemize}
\item {\ttfamily wrench\+::\+Simulation\+::get\+Energy\+Consumed\+By\+Host()}
\item {\ttfamily wrench\+::\+Simulation\+::get\+Total\+Energy\+Consumed()}
\item {\ttfamily \hyperlink{classwrench_1_1_simulation_adf4675a8c9c62c93bfbd3dc4a4e46556}{wrench\+::\+Simulation\+::set\+Pstate()}}
\item {\ttfamily \hyperlink{classwrench_1_1_simulation_a5e8d5b963d2278c79b49a0ed7db2b933}{wrench\+::\+Simulation\+::get\+Numberof\+Pstates()}}
\item {\ttfamily \hyperlink{classwrench_1_1_simulation_a3d204b229feec1eee4f1e82d92490d81}{wrench\+::\+Simulation\+::get\+Current\+Pstate()}}
\item {\ttfamily \hyperlink{classwrench_1_1_simulation_afdf2ae84f6b3c8b51c5189199bebb52e}{wrench\+::\+Simulation\+::get\+Min\+Power\+Consumption()}}
\item {\ttfamily \hyperlink{classwrench_1_1_simulation_ae76b92ce868c6e6c1683377d869a5b34}{wrench\+::\+Simulation\+::get\+Max\+Power\+Consumption()}}
\item {\ttfamily \hyperlink{classwrench_1_1_simulation_abb75fd040236995186d9ad45434fe069}{wrench\+::\+Simulation\+::get\+List\+Of\+Pstates()}}
\end{DoxyItemize}

{\bfseries Note\+:} The energy plugin is N\+OT enabled by default in W\+R\+E\+N\+CH simulation. To enable the plugin, the {\ttfamily -\/-\/activate-\/energy} command line option should be provided when running a simulator. 